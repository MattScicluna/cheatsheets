\documentclass[a4paper, columns=2, hidelinks]{cheatsheet}
\usepackage{graphicx}
\usepackage{tikz}
\usetikzlibrary{positioning}
\usetikzlibrary{shadows}
\usetikzlibrary{shapes.misc}


% formatting
\usepackage{fullpage}
\usepackage{tasks}

% Theorems, Definitions
\usepackage{amsthm, amsmath}
\usepackage[most]{tcolorbox}
\usepackage{framed}
\usepackage{adjustbox}

% section
\newtcbtheorem{Section}{Section}{
	enhanced,
	sharp corners,
	boxrule=0pt,
	toprule=1pt,bottomrule=0pt,
	left=0.2cm,right=0.2cm,top=0.2cm,
	titlerule=0.5em,
	toptitle=0.1cm,
	bottomtitle=-0.1cm,
	attach boxed title to top left={yshift=-2mm},
	colback=white,
	colframe=black!50,
	fonttitle=\bfseries,
	boxed title style={
		size=small,
		colback=black!50,
		colframe=black!50,
	} 
}{thm}

% Proposition
\newtheoremstyle{propstyle}
{\topsep}%
{\topsep}%
{\itshape}%
{}%
{\bfseries}
{}
{.5em}%
{
{\thmname{#1}~\thmnumber{#2}\thmnote{ (#3)}}
}
\theoremstyle{propstyle}
\newtheorem{proposition}{Prop}

% Definition
\newtheoremstyle{defstyle}
{\topsep}%
{\topsep}%
{\itshape}%
{}%
{\bfseries}
{}
{.5em}%
{
{\thmname{#1}~\thmnumber{#2}\thmnote{ (#3)}}
}
\theoremstyle{defstyle}

\newtheorem{definition}{Def}


%  Make title
\title{Probabilistic Graphical Models Notes}

%  Make narrow margins
\newgeometry{margin=0.15in}

% For lists
\usepackage{enumitem}

\begin{document}

\onecolumn
\setlength{\abovedisplayskip}{0pt}
\setlength{\belowdisplayskip}{0pt}
\setlength{\abovedisplayshortskip}{0pt}
\setlength{\belowdisplayshortskip}{0pt}
\setlength{\jot}{0pt}

\maketitle

\begin{Section}{Background}{}
\begin{definition}[Conditional Independence]
Given RVs $X, Y, Z$ we say that $X 	\perp Y \mid Z$ iff $P(X, Y| Z) = P(X \mid Z)P(Y \mid Z)$
\end{definition}
\bfseries  \normalfont Useful properties:
		\begin{tasks}(2)
			\task Symmetry: $X \perp Y \mid Z \Rightarrow Y \perp X \mid Z$
			\task Decomposition: $X \perp Y,W \mid Z \Rightarrow 
			\begin{cases}
			X \perp Y \mid Z \\
			X \perp W \mid Y
			\end{cases}$ 
			\task Weak Union: $X \perp Y,W \mid Z \Rightarrow 	
			\begin{cases}
			X \perp Y \mid Z, W \\
			X \perp W \mid Z, Y
			\end{cases}$
			\task Contraction: $\begin{cases}
			X \perp Y \mid Z,W \\
			X \perp W \mid Z
			\end{cases}
			\Rightarrow X \perp Y, W \mid Z
			$
		\end{tasks}
An ordering \(I: v \mapsto \{1,\cdots,n\}\) is \textbf{Topological} iff \(j \in \pi_i \Rightarrow I(j) < I(i)\) \emph{i.e. parents always come before their children}
If \(G\) is a DAG, then \(\exists\) a Topological Ordering on \(G\)
A \textbf{Forest} is a Graph s.t. each node has at most one parent
A \textbf{Tree} is a Forest if it is connected
\end{Section}

\begin{Section}{Directed Graphical Models}{}
\begin{definition}[Factorization Property of DAGs]
	Given DAG \(G=(V,E)\)\\
	\(\mathcal{L}(G)=\left\{p \text{ is a dist over } x_v : \exists \text{ factors } f_i \text{ s.t. } p(x_v) = \prod_{i=1}^n f_i(x_i; x_{\pi_i}) \text{ and } f_i \text{ satisfies: } \begin{array}{lll}
	f_i>0 \\
	\sum_{x_i}f_i(x_i; x_{\pi_i}) = 1 \\
	f_i: Dom(x_i)^2 \mapsto [0,1]
	\end{array}
	\right\} \)
\end{definition}

\begin{proposition}[Leaf Plucking Property]
	if \(n\) is a leaf of \(G\), \(p(x_v)\in \mathcal{L}\left(G-\{n\}\right)\)
\end{proposition}
\begin{proof}
	$P(x_v)=p(x_{1:n-1},x_n)=f_n(x_n;x_{\pi_n})\prod_{i\ne n} f_i(x_i,x_{\pi_i})$. Next marginalizing out \(x_n\) and using that it is a leaf, we have:
	\begin{align*}p(x_{1:n-1})=\underbrace{\sum_{x_n}f_n(x_n;x_{\pi_n})}_{\text{sums to 1}}\underbrace{\prod_{i=1}^{n-1} f_i(x_i;x_{\pi_i})}_{\text{Does not contain } x_n} = \prod_{i=1}^{n-1} f_i(x_i;x_{\pi_i})
	\end{align*}
	So $p(x_v) \in \mathcal{L}\left(G-\{n\}\right)$ as needed.
\end{proof}
\begin{proposition}[Factors are Conditional PMFs]
	Let \(p \in \mathcal{L}(G)\) and $\{f_j\}$ be a factorization, then $\forall i, P(x_i | x_{\pi_i}) = f_i(x_i;x_{\pi_i})$
\end{proposition}
\begin{proof}
	WLOG let $\{1,\cdots,n\}$ be a Topological Ordering and use Theorem 1 to get that \(p(x_{1:i}) \in \mathcal{L}(G-\{i+1,\cdots,n\})\), and so \(p(x_{1:i})=\underbrace{\prod_{j=1}^{i-1} f_j(x_j;x_{\pi_j})}_{\text{Call this } g(x_{1:i-1})} f_i(x_i;x_{\pi_i})\). We partition \(\{1:i\}\) as \( \{i\}\cup \pi_i \cup A\) and get that:
	\begin{align*}
	p(x_i \mid x_{\pi_i})= \frac{\sum_{x_A}f_i(x_i;x_{\pi_i})g(x_{1:i-1})}{\sum_{x_A}\sum_{x'_i}f_i(x'_i;x'_{\pi_i})g(x_{1:i-1})}=\frac{f_i(x_i;x_{\pi_i})\sum_{x_A}g(x_{1:i-1})}{\sum_{x'_i}f_i(x'_i;x'_{\pi_i})\sum_{x_A}g(x_{1:i-1})}=f_i(x_i;x_{\pi_i})
	\end{align*}
\end{proof}
Note: adding edges adds more distributions i.e. \(E \subseteq E'\) and \(G' = (V, E')\) then \(\mathcal{L}(G) \subseteq \mathcal{L}(G')\)
\begin{proposition}[] 
	\( p \in \mathcal{L}(G) \iff x_i \perp x_{nd(i)} \mid \pi_i\)
\end{proposition}
\begin{proof}
	(\(\Rightarrow\))
	(\(\Leftarrow\))
\end{proof}
\end{Section}




\newpage

\bibliography{prob_cs}
\bibliographystyle{ieeetr}



\end{document}